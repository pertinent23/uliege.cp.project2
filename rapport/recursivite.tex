% !TEX root = ./main.tex
%%%%%%%%%%%%%%%%%%%%%%%%%%%%%%%%%%%%%%%%%%%%%%%%%%%%%%%%%%%%%%%%%%%%%%%%%%%%%%%%%%%%%%%%%%
% Justifiez, dans cette section, chacune de vos implémentations récursives à l'aide des  %
% 3 étapes vues au cours (cfr. Chap. 4)                                                  %
%%%%%%%%%%%%%%%%%%%%%%%%%%%%%%%%%%%%%%%%%%%%%%%%%%%%%%%%%%%%%%%%%%%%%%%%%%%%%%%%%%%%%%%%%%
\section{Implémentations Récursives}\label{recursivite}
%%%%%%%%%%%%%%%%%%%%%%%%%%%%%%%%%%%%%

\quad Nous avons implémenté recursivement la fonction $in\_list(head, value)$, qui permet de vérifier
si une value se trouve parmis les élements précédents d'un Noeud, et va retourner 1 dans le cas favorable
sinon 0.

\begin{itemize}
    \item head: est le dernier Noeud d'une liste d'oublement chainée
    \item value: est la valeur recherchée dans la liste
\end{itemize}

\subsection{Conception:}

\begin{itemize}
    \item Cas de base: ici deux cas sont possibles
        \indent
        \begin{itemize}
            \item $head == NULL$ dans ce cas l'élément n'est pas dans la liste on retourne 0
            \item $head \ne NULL \land head\to value == value$ dans ce cas on retourne 1
        \end{itemize}
    \item Cas non récursif: $head \ne NULL \land head\to value \ne value$
\end{itemize}

\subsection{Implémentation:}

\begin{lstlisting}
static unsigned int in_list(struct Node_t *last, Region *value)
{
    struct Node_t *head = last;
    if (head != NULL)
    {
        if (head->value == value)
            return 1;
        return in_list(head->prev, value);
    }
    
    return 0;
}
\end{lstlisting}
