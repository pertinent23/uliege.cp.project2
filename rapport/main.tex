%%%%%%%%%%%%%%%%%%%%%%%%%%%%%%%%%%%%%%%%%%%%%%%%%%%%%%%%%%%%%%%%%%%%%%%%%%%%%%%%%%%%%%%%%%
% Ceci est le fichier principal du template template à utiliser pour les rapports du     %
% projet 2 (TAD) d'INFO0947.                                                             %
%                                                                                        %
% Vous devez décommenter et compléter les commandes introduites plus bas (intitule, ...) %
% avant de pouvoir compiler le fichier LaTeX.  Pensez à configurer votre Makefile en     %
% conséquence.                                                                           %
%                                                                                        %
% Le contenu et la structure du rapport sont imposés.  Vous devez compléter les          %
% différents fichiers .tex inclus dans ce fichier avec votre production.                 %
%%%%%%%%%%%%%%%%%%%%%%%%%%%%%%%%%%%%%%%%%%%%%%%%%%%%%%%%%%%%%%%%%%%%%%%%%%%%%%%%%%%%%%%%%%

% !TEX root = ./main.tex
% !TEX engine = latexmk -pdf
% !TEX buildOnSave = true
\documentclass[a4paper, 11pt, oneside]{article}

\usepackage[utf8]{inputenc}
\usepackage[T1]{fontenc}
\usepackage[french]{babel}
\usepackage{array}
\usepackage{shortvrb}
\usepackage{listings}
\usepackage[fleqn]{amsmath}
\usepackage{amsfonts}
\usepackage{fullpage}
\usepackage{enumerate}
\usepackage{graphicx}             % import, scale, and rotate graphics
\usepackage{subfigure}            % group figures
\usepackage{alltt}
\usepackage{url}
\usepackage{indentfirst}
\usepackage{eurosym}
\usepackage{listings}
\usepackage{color}
\usepackage[table,xcdraw,dvipsnames]{xcolor}
\usepackage{multirow}

\graphicspath{ {./images/} }

\definecolor{mygray}{rgb}{0.5,0.5,0.5}
\newcommand{\coms}[1]{\textcolor{MidnightBlue}{#1}}

\lstset{
    language=C, % Utilisation du langage C
    commentstyle={\color{MidnightBlue}}, % Couleur des commentaires
    frame=single, % Entoure le code d'un joli cadre
    rulecolor=\color{black}, % Couleur de la ligne qui forme le cadre
    stringstyle=\color{RawSienna}, % Couleur des chaines de caractères
    numbers=left, % Ajoute une numérotation des lignes à gauche
    numbersep=5pt, % Distance entre les numérots de lignes et le code
    numberstyle=\tiny\color{mygray}, % Couleur des numéros de lignes
    basicstyle=\tt\footnotesize,
    tabsize=3, % Largeur des tabulations par défaut
    keywordstyle=\tt\bf\footnotesize\color{Sepia}, % Style des mots-clés
    extendedchars=true,
    captionpos=b, % sets the caption-position to bottom
    texcl=true, % Commentaires sur une ligne interprétés en Latex
    showstringspaces=false, % Ne montre pas les espace dans les chaines de caractères
    escapeinside={(>}{<)}, % Permet de mettre du latex entre des <( et )>.
    inputencoding=utf8,
    literate=
  {á}{{\'a}}1 {é}{{\'e}}1 {í}{{\'i}}1 {ó}{{\'o}}1 {ú}{{\'u}}1
  {Á}{{\'A}}1 {É}{{\'E}}1 {Í}{{\'I}}1 {Ó}{{\'O}}1 {Ú}{{\'U}}1
  {à}{{\`a}}1 {è}{{\`e}}1 {ì}{{\`i}}1 {ò}{{\`o}}1 {ù}{{\`u}}1
  {À}{{\`A}}1 {È}{{\`E}}1 {Ì}{{\`I}}1 {Ò}{{\`O}}1 {Ù}{{\`U}}1
  {ä}{{\"a}}1 {ë}{{\"e}}1 {ï}{{\"i}}1 {ö}{{\"o}}1 {ü}{{\"u}}1
  {Ä}{{\"A}}1 {Ë}{{\"E}}1 {Ï}{{\"I}}1 {Ö}{{\"O}}1 {Ü}{{\"U}}1
  {â}{{\^a}}1 {ê}{{\^e}}1 {î}{{\^i}}1 {ô}{{\^o}}1 {û}{{\^u}}1
  {Â}{{\^A}}1 {Ê}{{\^E}}1 {Î}{{\^I}}1 {Ô}{{\^O}}1 {Û}{{\^U}}1
  {œ}{{\oe}}1 {Œ}{{\OE}}1 {æ}{{\ae}}1 {Æ}{{\AE}}1 {ß}{{\ss}}1
  {ű}{{\H{u}}}1 {Ű}{{\H{U}}}1 {ő}{{\H{o}}}1 {Ő}{{\H{O}}}1
  {ç}{{\c c}}1 {Ç}{{\c C}}1 {ø}{{\o}}1 {å}{{\r a}}1 {Å}{{\r A}}1
  {€}{{\euro}}1 {£}{{\pounds}}1 {«}{{\guillemotleft}}1
  {»}{{\guillemotright}}1 {ñ}{{\~n}}1 {Ñ}{{\~N}}1 {¿}{{?`}}1
}
\newcommand{\tablemat}{~}

%%%%%%%%%%%%%%%%% TITRE %%%%%%%%%%%%%%%%
% Complétez et décommentez les définitions de macros suivantes :
\newcommand{\intitule}{FLAMME OLYMPIQUE}
\newcommand{\GrNbr}{26}
\newcommand{\PrenomUN}{Franck Duval}
\newcommand{\NomUN}{HEUBA BATOMEN}
\newcommand{\PrenomDEUX}{Bilali}
\newcommand{\NomDEUX}{ASSALNI}

\renewcommand{\tablemat}{\tableofcontents}

%%%%%%%% ZONE PROTÉGÉE : MODIFIEZ UNE DES DIX PROCHAINES %%%%%%%%
%%%%%%%%            LIGNES POUR PERDRE 2 PTS.            %%%%%%%%
\title{INFO0947: \intitule}
\author{Groupe \GrNbr: \PrenomUN~\textsc{\NomUN}, \PrenomDEUX~\textsc{\NomDEUX}}
\date{}
\begin{document}

\maketitle
\newpage
\tablemat
\newpage

%%%%%%%%%%%%%%%% RAPPORT %%%%%%%%%%%%%%%

% Inclusion des différentes sections

% !TEX root = ./main.tex
%%%%%%%%%%%%%%%%%%%%%%%%%%%%%%%%%%%%%%%%%%%%%%%%%%%%%%%%%%%%%%%%%%%%%%%%%%%%%%%%%%%%%%%%%%
% Rédigez ici l'introduction de votre rapport.                                           %
%%%%%%%%%%%%%%%%%%%%%%%%%%%%%%%%%%%%%%%%%%%%%%%%%%%%%%%%%%%%%%%%%%%%%%%%%%%%%%%%%%%%%%%%%%
\section{Introduction}\label{introduction}
%%%%%%%%%%%%%%%%%%%%%%%

\quad\quad la flamme Olympique arrive dans le pays hôte c’est l’esprit des
Jeux qui débarque. Avant la cérémonie d'ouverture, la flamme, portée
par une multitude de relayeurs, réalise un parcours jusqu’à la ville hôte des Jeux.
Ainsi, elle devra parcourir un ensemble de villes, constituant ainsi un itinéraire 
jusqu’à la ville pour la cérémonie d'ouverture.
\quad Ainsi, le travail que nous avons réalisé à consisté à numériser ce parcours de
ville en ville, region en region.

% !TEX root = ./main.tex
%%%%%%%%%%%%%%%%%%%%%%%%%%%%%%%%%%%%%%%%%%%%%%%%%%%%%%%%%%%%%%%%%%%%%%%%%%%%%%%%%%%%%%%%%%
% Dans cette section, spécifiez formellement vos TADs (syntaxe et sémantique)            %
% 1 sous-section/TAD                                                                     %
% N'oubliez pas de justifier la complétude de vos TADs                                   %
%%%%%%%%%%%%%%%%%%%%%%%%%%%%%%%%%%%%%%%%%%%%%%%%%%%%%%%%%%%%%%%%%%%%%%%%%%%%%%%%%%%%%%%%%%
\section{Spécifications Abstraites}\label{tad}
%%%%%%%%%%%%%%%%%%%%%%%%%%%%%%%%%%%

Nous avons principalement deux types abstraits de données:
\begin{itemize}
  \item Region
  \item ItineraireFlame
\end{itemize}

\subsection{TAD Region}
%%%%%%%%%%%%%%%%%%%%%%%%
\subsubsection{Syntaxe}
%%%%%%%%%%%%%%%%%%%%%%%
\begin{description}
  \item[Type:]
    \begin{description}
      \item Region
    \end{description}
  \item[Utilise:]\indent
    \begin{itemize}
      \item Integer
      \item String
      \item Double
    \end{itemize}
  \item[Opérations:]\indent
    \begin{itemize}
      \item create: Double $\times$ Double $\times$ String $\to$ Region
      \item get\_x: Region $\to$ Double
      \item get\_y: Region $\to$ Double
      \item get\_nb\_people: Region $\to$ Integer
      \item get\_headquater: Region $\to$ String
      \item get\_name: Region $\to$ String
      \item get\_speciality: Region $\to$ String
      \item distance: Region $\times$ Region $\to$ Double
      \item set\_x: Region $\times$ Double $\to$ Region
      \item set\_y: Region $\times$ Double $\to$ Region
      \item set\_headquater: Region $\times$ String $\to$ Region
      \item set\_speciality: Region $\times$ String $\to$ Region
      \item set\_nb\_people: Region $\times$ Integer $\to$ Region
      \item destroy: Region $\to$ $\emptyset$
    \end{itemize}
\end{description}

\subsubsection{Sémantique}
%%%%%%%%%%%%%%%%%%%%%%%%%%
\begin{description}
  \item[Préconditions:]\indent
    \begin{description}
      \item $\forall j \in$ Integer, $\forall k\in$ Region
      \item $\forall j \geq 0$, set\_nb\_people(k, j)
    \end{description}
  \item[Axiomes:]\indent
    \begin{description}
      \item $\forall r \in$ Region, $\forall i \in$ Double, $\forall j \in$ Integer,  $\forall s \in$ String
      \item $get\_x(set\_x(r, i)) = i$
      \item $get\_y(set\_y(r, i)) = i$
      \item $get\_speciality(set\_speciality(r, s)) = s$
      \item $get\_headquater(set\_headquater(r, s)) = s$
    \end{description}
\end{description}

\addvspace{50px}

\begin{raggedleft}
\begin{tabular}{ c|lccccc }
                                &                          & \multicolumn{4}{c}{Opérations Internes} \\
  \hline
  \multirow{6}{*}{Observateurs} &                          & $create(\cdot)$ & $set\_x(\cdot)$ & $set\_y(\cdot)$ & $set\_headquater(\cdot)$ \\
                                & $get\_x(\cdot)$          & \checkmark      & \checkmark      & \checkmark      & \checkmark               \\
                                & $get\_y(\cdot)$          & \checkmark      & \checkmark      & \checkmark      & \checkmark               \\
                                & $get\_headquater(\cdot)$ & \checkmark      & \checkmark      & \checkmark      & \checkmark               \\
                                & $get\_name(\cdot)$       & \checkmark      & \checkmark      & \checkmark      & \checkmark               \\
                                & $get\_speciality(\cdot)$ & \checkmark      & \checkmark      & \checkmark      & \checkmark               \\
\end{tabular}
\end{raggedleft}

\addvspace{50px}

\begin{raggedleft}
\begin{tabular}{ c|lccccc }
                                &                          & \multicolumn{2}{c}{Opérations Internes} \\
  \hline
  \multirow{6}{*}{Observateurs} &                          & $set\_nb\_people(\cdot)$ & $set\_speciality(\cdot)$ & $destroy(\cdot)$ \\
                                & $get\_x(\cdot)$          & \checkmark               & \checkmark               & $\emptyset$      \\
                                & $get\_y(\cdot)$          & \checkmark               & \checkmark               & $\emptyset$      \\
                                & $get\_headquater(\cdot)$ & \checkmark               & \checkmark               & $\emptyset$      \\
                                & $get\_name(\cdot)$       & \checkmark               & \checkmark               & $\emptyset$      \\
                                & $get\_speciality(\cdot)$ & \checkmark               & \checkmark               & $\emptyset$      \\
\end{tabular}
\end{raggedleft}


\pagebreak

%%%%%%%%%%%%%%%%%%%%%%%%%%%%%%%%%%%%%%%%%%%%%%%%%%%%%%%%%%%%%%%%%%%%%%%%%%%%%%%
%%%%%%%%%%%%%%%%%%%%%%%%%%%%%%%%%%%%%%%%%%%%%%%%%%%%%%%%%%%%%%%%%%%%%%%%%%%%%%%
%%%%%%%%%%%%%%%%%%%%%%%%%%%%%%%%%%%%%%%%%%%%%%%%%%%%%%%%%%%%%%%%%%%%%%%%%%%%%%%
%%%%%%%%%%%%%%%%%%%%%%%%%%%%%%%%%%%%%%%%%%%%%%%%%%%%%%%%%%%%%%%%%%%%%%%%%%%%%%%

\subsection{TAD ItineraireFlame}

\subsubsection{Syntaxe}

\begin{description}
  \item[Type:]
    \begin{description}
      \item ItineraireFlame
    \end{description}
  \item[Utilise:]\indent
    \begin{itemize}
      \item Region
      \item Boolean
      \item Integer
    \end{itemize}
  \item[Opérations:]\indent
    \begin{itemize}
      \item create: Region $\times$ Region $\to$ ItineraireFlame
      \item is\_circuit: ItineraireFlame $\to$ Boolean
      \item count\_region: ItineraireFlame $\to$ Integer
      \item count\_people: ItineraireFlame $\to$ Integer
      \item add\_region: ItineraireFlame $\times$ Region $\to$ ItineraireFlame
      \item remove\_region: ItineraireFlame $\times$ Region $\to$ ItineraireFlame
      \item destroy: ItineraireFlame $\to$ $\emptyset$
    \end{itemize}
\end{description}

\subsubsection{Sémantique}
%%%%%%%%%%%%%%%%%%%%%%%%%%
\begin{description}
  \item[Préconditions:]\indent
    \begin{description}
      \item $\forall i, j \in$ Region
      \item $\forall i, j$, create(i, j)
    \end{description}
  \item[Axiomes:]\indent
    \begin{description}
      \item $\forall r_0, r \in$ Region, $\forall j \in$ Integer, $\forall k \in$ ItineraireFlame
      \item $count\_region(add\_region(k, r)) = count\_region(k)+1$
      \item $is\_circuit(add\_region(create(r_0, r), r)) = True$
      \item $is\_circuit(add\_region(create(r_0, r), r_0)) = True$
      \item $is\_circuit(create(r_0, r)) = False$
    \end{description}
\end{description}

\begin{center}
\begin{tabular}{ c|lcccc }
                              & \multicolumn{5}{c}{Opérations Internes} \\
\hline
\multirow{4}{*}{Observateurs} &                           & $create(\cdot)$ & $add\_region(\cdot)$ & $remove\_region(\cdot)$ & $destroy(\cdot)$  \\
                              & $is\_circuit(\cdot)$      & \checkmark      & \checkmark           & \checkmark              & $\emptyset$       \\
                              & $count\_region(\cdot)$    & \checkmark      & \checkmark           & \checkmark              & $\emptyset$       \\
                              & $count\_people(\cdot)$    & \checkmark      & \checkmark           & \checkmark              & $\emptyset$       \\
\end{tabular}
\end{center}


\pagebreak

% !TEX root = ./main.tex
%%%%%%%%%%%%%%%%%%%%%%%%%%%%%%%%%%%%%%%%%%%%%%%%%%%%%%%%%%%%%%%%%%%%%%%%%%%%%%%%%%%%%%%%%%
% Dans cette section, spécifiez formellement chacune des fonctionalités.                 %
%%%%%%%%%%%%%%%%%%%%%%%%%%%%%%%%%%%%%%%%%%%%%%%%%%%%%%%%%%%%%%%%%%%%%%%%%%%%%%%%%%%%%%%%%%
\section{Specifications}\label{specifications}
%%%%%%%%%%%%%%%%%%%%%%%%


% !TEX root = ./main.tex
%%%%%%%%%%%%%%%%%%%%%%%%%%%%%%%%%%%%%%%%%%%%%%%%%%%%%%%%%%%%%%%%%%%%%%%%%%%%%%%%%%%%%%%%%%
% Dans cette section, indiquez et décrivez tous les Invariants nécessaires.              %
%                                                                                        %
% Pour chaque SP nécessitant un Invariant (une sous-section/SP):                         %
% - Donnez l'Invariant Graphique                                                         %
% - Donnez l'Invariant Formel correspondant à l'Invariant Graphique                      %
% Pensez à utiliser les notations définies précédemment.                                 %
%%%%%%%%%%%%%%%%%%%%%%%%%%%%%%%%%%%%%%%%%%%%%%%%%%%%%%%%%%%%%%%%%%%%%%%%%%%%%%%%%%%%%%%%%%
\section{Invariants}\label{invariants}
%%%%%%%%%%%%%%%%%%%%

\begin{itemize}

\item[]
\begin{itemize}
    \item[]$In(a, b)$: Permet de vérifier si b est dans a
        \indent
        \begin{description}
            \item a: Est un tableau de Region de taille ARRAY\_SIZE
            \item b: b est de type Region
        \end{description} 
    \item[]$CountRegion(map)$: Compte le nombre de region différentes dans map
        \indent
        \begin{description}
            \item map: Est un tableau de Region de taille ARRAY\_SIZE
        \end{description} 
    \item[]$CountHabitant(map)$: Compte le nombre d'habitant des regions différentes de map
        \indent
        \begin{description}
            \item map: Est un tableau de Region de taille ARRAY\_SIZE
        \end{description} 
\end{itemize}

\end{itemize}

%%%%%%%%%%%%%%%%%%%%%%%%%%%%%%%%%%%%%%%%%%%%%%%%%%%%%%%%%%%%%%%%%%%%%%%%%%%%%%%%%%%%%%%%%%%%
%%%%%%%%%%%%%%%%%%%%%%%%%%%%%%%%%%%%%%%%%%%%%%%%%%%%%%%%%%%%%%%%%%%%%%%%%%%%%%%%%%%%%%%%%%%%
%%%%%%%%%%%%%%%%%%%%%%%%%%%%%%%%%%%%%%%%%%%%%%%%%%%%%%%%%%%%%%%%%%%%%%%%%%%%%%%%%%%%%%%%%%%%
%%%%%%%%%%%%%%%%%%%%%%%%%%%%%%%%%%%%%%%%%%%%%%%%%%%%%%%%%%%%%%%%%%%%%%%%%%%%%%%%%%%%%%%%%%%%

\subsection{Recherche d'une region sur l'itinéraire: $search(map, region)$}

\begin{itemize}
    \item region: de type Region
    \item map[0, indicator]: est un tableau de region
\end{itemize}

\subsubsection{Invariant Graphique}

\addvspace{40px}
\begin{center}
    \includegraphics[scale=0.5]{search-region.png}
\end{center}
\addvspace{40px}

\subsubsection{Invariant Formel}

INV $\equiv$ $map=map_0 \land -1 \le i < indicator$


%%%%%%%%%%%%%%%%%%%%%%%%%%%%%%%%%%%%%%%%%%%%%%%%%%%%%%%%%%%%%%%%%%%%%%%%%%%%%%%%%%%%%%%%%%%%
%%%%%%%%%%%%%%%%%%%%%%%%%%%%%%%%%%%%%%%%%%%%%%%%%%%%%%%%%%%%%%%%%%%%%%%%%%%%%%%%%%%%%%%%%%%%
%%%%%%%%%%%%%%%%%%%%%%%%%%%%%%%%%%%%%%%%%%%%%%%%%%%%%%%%%%%%%%%%%%%%%%%%%%%%%%%%%%%%%%%%%%%%
%%%%%%%%%%%%%%%%%%%%%%%%%%%%%%%%%%%%%%%%%%%%%%%%%%%%%%%%%%%%%%%%%%%%%%%%%%%%%%%%%%%%%%%%%%%%

\subsection{Compter le nombre de region sur l'itinéraire: $count\_region(map)$}

\begin{itemize}
    \item map[0, indicator]: est un tableau de region
\end{itemize}

\subsubsection{Invariant Graphique}

\addvspace{40px}
\begin{center}
    \includegraphics[scale=0.5]{count-region.png}
\end{center}
\addvspace{40px}

\subsubsection{Invariant Formel}

INV $\equiv$ $map=map_0 \land 0 \le i \le indicator \land compteur=CountRegion(map[0, i])$

%%%%%%%%%%%%%%%%%%%%%%%%%%%%%%%%%%%%%%%%%%%%%%%%%%%%%%%%%%%%%%%%%%%%%%%%%%%%%%%%%%%%%%%%%%%%
%%%%%%%%%%%%%%%%%%%%%%%%%%%%%%%%%%%%%%%%%%%%%%%%%%%%%%%%%%%%%%%%%%%%%%%%%%%%%%%%%%%%%%%%%%%%
%%%%%%%%%%%%%%%%%%%%%%%%%%%%%%%%%%%%%%%%%%%%%%%%%%%%%%%%%%%%%%%%%%%%%%%%%%%%%%%%%%%%%%%%%%%%
%%%%%%%%%%%%%%%%%%%%%%%%%%%%%%%%%%%%%%%%%%%%%%%%%%%%%%%%%%%%%%%%%%%%%%%%%%%%%%%%%%%%%%%%%%%%

\subsection{Compter le nombre d'habitant de l'itinéraire: $count\_resident(map)$}

\subsubsection{Invariant Graphique}

\addvspace{40px}
\begin{center}
    \includegraphics[scale=0.5]{count-habitant}
\end{center}
\addvspace{40px}

\subsubsection{Invariant Formel}

INV $\equiv$ $map=map_0 \land 0 \le i \le indicator \land compteur=CountHabitant(map[0, i])$

%%%%%%%%%%%%%%%%%%%%%%%%%%%%%%%%%%%%%%%%%%%%%%%%%%%%%%%%%%%%%%%%%%%%%%%%%%%%%%%%%%%%%%%%%%%%
%%%%%%%%%%%%%%%%%%%%%%%%%%%%%%%%%%%%%%%%%%%%%%%%%%%%%%%%%%%%%%%%%%%%%%%%%%%%%%%%%%%%%%%%%%%%
%%%%%%%%%%%%%%%%%%%%%%%%%%%%%%%%%%%%%%%%%%%%%%%%%%%%%%%%%%%%%%%%%%%%%%%%%%%%%%%%%%%%%%%%%%%%
%%%%%%%%%%%%%%%%%%%%%%%%%%%%%%%%%%%%%%%%%%%%%%%%%%%%%%%%%%%%%%%%%%%%%%%%%%%%%%%%%%%%%%%%%%%%

\subsection{Vérifier si un itinéraire est un circuit: $is\_circuit(map)$}

\subsubsection{Invariant Graphique}

\addvspace{40px}
\begin{center}
    \includegraphics[scale=0.5]{is_circuit.png}
\end{center}
\addvspace{40px}

\subsubsection{Invariant Formel}

INV $\equiv$ $map=map_0 \land 0 \le i \le indicator \land is\_circuit=In(map[0, i-1], map[i])$

% !TEX root = ./main.tex
%%%%%%%%%%%%%%%%%%%%%%%%%%%%%%%%%%%%%%%%%%%%%%%%%%%%%%%%%%%%%%%%%%%%%%%%%%%%%%%%%%%%%%%%%%
% Justifiez, dans cette section, chacune de vos implémentations récursives à l'aide des  %
% 3 étapes vues au cours (cfr. Chap. 4)                                                  %
%%%%%%%%%%%%%%%%%%%%%%%%%%%%%%%%%%%%%%%%%%%%%%%%%%%%%%%%%%%%%%%%%%%%%%%%%%%%%%%%%%%%%%%%%%
\section{Implémentations Récursives}\label{recursivite}
%%%%%%%%%%%%%%%%%%%%%%%%%%%%%%%%%%%%%


% !TEX root = ./main.tex
%%%%%%%%%%%%%%%%%%%%%%%%%%%%%%%%%%%%%%%%%%%%%%%%%%%%%%%%%%%%%%%%%%%%%%%%%%%%%%%%%%%%%%%%%%
% Dans cette section, vous devez étudier complètement la complexité de votre code.       %
% Soyez le plus formel (i.e., mathématique) possible.                                    %
%%%%%%%%%%%%%%%%%%%%%%%%%%%%%%%%%%%%%%%%%%%%%%%%%%%%%%%%%%%%%%%%%%%%%%%%%%%%%%%%%%%%%%%%%%
\section{Complexité}\label{complexite}
%%%%%%%%%%%%%%%%%%%%

\quad Le projet est divisé en deux partie, Region et ItineraireFlame.

\subsection{Partie Region:}

Dans cette partie nous n'avons que des accésseurs et des mutateurs, ainsi, ils fonctionnent
en $O(1)$ sur les champs de la structure car aucun tableau ou liste est utilisé.

\subsection{Partie ItineraireFlame:}

\subsubsection{Create}

\begin{lstlisting}
struct ItineraireFlame_t *new_itineraireflame(Region *start, Region *end)
{
    assert(start != NULL && end != NULL);

    //$T_1$
    if (start == end)
        return NULL;

    //$T_2$
    struct ItineraireFlame_t *way = malloc(sizeof(struct ItineraireFlame_t));

    //$T_3$
    if (way == NULL)
        return NULL;

    //$T_4$
    way->map = malloc(sizeof(Region *) * ARRAY_SIZE);

    //$T_5$
    if (way->map == NULL)
    {
        free(way);
        return NULL;
    }

    //$T_6$
    way->departure = start;
    way->arrival = end;
    way->indicator = 0;
    
    return way;
}
\end{lstlisting}

La complexité ici est égale à la somme de $T_1$ à $T_6$

$T(.) = \sum_{i=1}^{6} T_i$

\begin{itemize}
    \item $T_1 = 1$
    \item $T_2 = 1$
    \item $T_3 = 1$
    \item $T_5 = 1$
    \item $T_6 = 1$
\end{itemize}

Ainsi $T(.) = 6 \rightarrow T(.) \in O(1)$

% !TEX root = ./main.tex
%%%%%%%%%%%%%%%%%%%%%%%%%%%%%%%%%%%%%%%%%%%%%%%%%%%%%%%%%%%%%%%%%%%%%%%%%%%%%%%%%%%%%%%%%%
% Dans cette section, décrivez comment vous avez implémenté les différents tests         %
% unitaires                                                                              %
% Pensez à justifier vos choix.                                                          %
%%%%%%%%%%%%%%%%%%%%%%%%%%%%%%%%%%%%%%%%%%%%%%%%%%%%%%%%%%%%%%%%%%%%%%%%%%%%%%%%%%%%%%%%%%
\section{Tests Unitaires}\label{tests}
%%%%%%%%%%%%%%%%%%%%%%%%%
Comme il y'a deux implémentations du module "itineraireflame.h", les tests unitaires ont étés
divisés en deux parties:
\begin{itemize}
    \item Les tests de l'implémentations avec les listes
    \item Les tests de l'implémentations avec les tableaux
\end{itemize}
Chacun de ces tests commencent par des tests sur le module "region.h", notaments:
\begin{itemize}
    \item La creation d'une region
    \item Les tests des Accésseurs
    \item Les tests des Mutateurs
\end{itemize}
Puis on fait des tests de calculs de distance entre les régions.

Une fois, les tests sur le module "region.h" terminés, on commencent les tests sur accésseurs basics 
du type ItineraireFlame, c'est à dire:
\begin{itemize}
    \item get\_coord\_x
    \item get\_coord\_y
    \item get\_region\_name
\end{itemize}
Ces tests passés, on test l'ajout et la suppréssion des regions et le comptage des regions.

% !TEX root = ./main.tex
%%%%%%%%%%%%%%%%%%%%%%%%%%%%%%%%%%%%%%%%%%%%%%%%%%%%%%%%%%%%%%%%%%%%%%%%%%%%%%%%%%%%%%%%%%
% Rédigez ici la conclusion de votre rapport.                                            %
%%%%%%%%%%%%%%%%%%%%%%%%%%%%%%%%%%%%%%%%%%%%%%%%%%%%%%%%%%%%%%%%%%%%%%%%%%%%%%%%%%%%%%%%%%
\section{Conclusion}\label{conclusion}
%%%%%%%%%%%%%%%%%%%%%


%%%%%%%%%%%%%%%%%%%% FIN DE LA ZONE PROTÉGÉE %%%%%%%%%%%%%%%%%%%%

\end{document}
