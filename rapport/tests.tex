% !TEX root = ./main.tex
%%%%%%%%%%%%%%%%%%%%%%%%%%%%%%%%%%%%%%%%%%%%%%%%%%%%%%%%%%%%%%%%%%%%%%%%%%%%%%%%%%%%%%%%%%
% Dans cette section, décrivez comment vous avez implémenté les différents tests         %
% unitaires                                                                              %
% Pensez à justifier vos choix.                                                          %
%%%%%%%%%%%%%%%%%%%%%%%%%%%%%%%%%%%%%%%%%%%%%%%%%%%%%%%%%%%%%%%%%%%%%%%%%%%%%%%%%%%%%%%%%%
\section{Tests Unitaires}\label{tests}
%%%%%%%%%%%%%%%%%%%%%%%%%
Comme il y'a deux implémentations du module "itineraireflame.h", les tests unitaires ont étés
divisés en deux parties:
\begin{itemize}
    \item Les tests de l'implémentations avec les listes
    \item Les tests de l'implémentations avec les tableaux
\end{itemize}
Chacun de ces tests commencent par des tests sur le module "region.h", notaments:
\begin{itemize}
    \item La creation d'une region
    \item Les tests des Accésseurs
    \item Les tests des Mutateurs
\end{itemize}
Puis on fait des tests de calculs de distance entre les régions.

Une fois, les tests sur le module "region.h" terminés, on commencent les tests sur accésseurs basics 
du type ItineraireFlame, c'est à dire:
\begin{itemize}
    \item get\_coord\_x
    \item get\_coord\_y
    \item get\_region\_name
\end{itemize}
Ces tests passés, on test l'ajout et la suppréssion des regions et le comptage des regions.