% !TEX root = ./main.tex
%%%%%%%%%%%%%%%%%%%%%%%%%%%%%%%%%%%%%%%%%%%%%%%%%%%%%%%%%%%%%%%%%%%%%%%%%%%%%%%%%%%%%%%%%%
% Rédigez ici la conclusion de votre rapport.                                            %
%%%%%%%%%%%%%%%%%%%%%%%%%%%%%%%%%%%%%%%%%%%%%%%%%%%%%%%%%%%%%%%%%%%%%%%%%%%%%%%%%%%%%%%%%%
\section{Conclusion}\label{conclusion}
%%%%%%%%%%%%%%%%%%%%%

En conclusion, il a été question tout au long de ce projet:
\begin{description}
    \item[D'abord de faire la conception des types abstraits utilisés]
    \indent
    \begin{itemize}
        \item Region
        \item ItineraireFlame
    \end{itemize}
    \item[Ensuite d'implémenter les types abstraits]  
\end{description}

Le type ItineraireFlame devant utiliser une structure de données itérative, nous avons dû
faire deux implémentations, donc une en utilisans des tableaux et l'autre en utilisans une liste
doublement chainées.
\newline
\quad Ainsi, nous avons procédé à l'établissement des invariants graphiques et formels pour chaque 
sous-problèmes.
\newline
\quad Enfin, nous avons dû procéder à la rédaction du rapport.