% !TEX root = ./main.tex
%%%%%%%%%%%%%%%%%%%%%%%%%%%%%%%%%%%%%%%%%%%%%%%%%%%%%%%%%%%%%%%%%%%%%%%%%%%%%%%%%%%%%%%%%%
% Dans cette section, vous devez étudier complètement la complexité de votre code.       %
% Soyez le plus formel (i.e., mathématique) possible.                                    %
%%%%%%%%%%%%%%%%%%%%%%%%%%%%%%%%%%%%%%%%%%%%%%%%%%%%%%%%%%%%%%%%%%%%%%%%%%%%%%%%%%%%%%%%%%
\section{Complexité}\label{complexite}
%%%%%%%%%%%%%%%%%%%%

\quad Le projet est divisé en deux partie, Region et ItineraireFlame.

\subsection{Partie Region:}

Dans cette partie nous n'avons que des accésseurs et des mutateurs, ainsi, ils fonctionnent
en $O(1)$ sur les champs de la structure car aucun tableau ou liste est utilisé.

\subsection{Partie ItineraireFlame:}

\subsubsection{Create}

\begin{lstlisting}
struct ItineraireFlame_t *new_itineraireflame(Region *start, Region *end)
{
    assert(start != NULL && end != NULL);

    //$T_1$
    if (start == end)
        return NULL;

    //$T_2$
    struct ItineraireFlame_t *way = malloc(sizeof(struct ItineraireFlame_t));

    //$T_3$
    if (way == NULL)
        return NULL;

    //$T_4$
    way->map = malloc(sizeof(Region *) * ARRAY_SIZE);

    //$T_5$
    if (way->map == NULL)
    {
        free(way);
        return NULL;
    }

    //$T_6$
    way->departure = start;
    way->arrival = end;
    way->indicator = 0;
    
    return way;
}
\end{lstlisting}

La complexité ici est égale à la somme de $T_1$ à $T_6$

$T(.) = \sum_{i=1}^{6} T_i$

\begin{itemize}
    \item $T_1 = 1$
    \item $T_2 = 1$
    \item $T_3 = 1$
    \item $T_5 = 1$
    \item $T_6 = 1$
\end{itemize}

Ainsi $T(.) = 6 \rightarrow T(.) \in O(1)$