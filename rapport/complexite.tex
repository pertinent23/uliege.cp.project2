% !TEX root = ./main.tex
%%%%%%%%%%%%%%%%%%%%%%%%%%%%%%%%%%%%%%%%%%%%%%%%%%%%%%%%%%%%%%%%%%%%%%%%%%%%%%%%%%%%%%%%%%
% Dans cette section, vous devez étudier complètement la complexité de votre code.       %
% Soyez le plus formel (i.e., mathématique) possible.                                    %
%%%%%%%%%%%%%%%%%%%%%%%%%%%%%%%%%%%%%%%%%%%%%%%%%%%%%%%%%%%%%%%%%%%%%%%%%%%%%%%%%%%%%%%%%%
\section{Complexité}\label{complexite}
%%%%%%%%%%%%%%%%%%%%

\quad Le projet est divisé en deux partie, Region et ItineraireFlame.

\subsection{Partie Region:}

Dans cette partie nous n'avons que des accésseurs et des mutateurs, ainsi, ils fonctionnent
en $O(1)$ sur les champs de la structure car aucun tableau ou liste est utilisé.

\subsection{Partie ItineraireFlame:}

\subsubsection{Create}

\begin{lstlisting}
struct ItineraireFlame_t *new_itineraireflame(Region *start, Region *end)
{
    assert(start != NULL && end != NULL);

    //$T_1$
    if (start == end)
        return NULL;

    //$T_2$
    struct ItineraireFlame_t *way = malloc(sizeof(struct ItineraireFlame_t));

    //$T_3$
    if (way == NULL)
        return NULL;

    //$T_4$
    way->map = malloc(sizeof(Region *) * ARRAY_SIZE);

    //$T_5$
    if (way->map == NULL)
    {
        free(way);
        return NULL;
    }

    //$T_6$
    way->departure = start;
    way->arrival = end;
    way->indicator = 0;
    
    return way;
}
\end{lstlisting}

La complexité ici est égale à la somme de $T_1$ à $T_6$

$T(.) = \sum_{i=1}^{6} T_i$

\begin{itemize}
    \item $T_1 = 1$
    \item $T_2 = 1$
    \item $T_3 = 1$
    \item $T_5 = 1$
    \item $T_6 = 1$
\end{itemize}

Ainsi $T(.) = 6 \rightarrow T(.) \in O(1)$

%%%%%%%%%%%%%%%%%%%%%%%%%%%%%%%%%%%%%%%%%%%%%%%%%%%%%%%%%%%%%%%%%%%%%%%%%%%%%%%%%%%%%%%%%%%%%%%%
%%%%%%%%%%%%%%%%%%%%%%%%%%%%%%%%%%%%%%%%%%%%%%%%%%%%%%%%%%%%%%%%%%%%%%%%%%%%%%%%%%%%%%%%%%%%%%%%
%%%%%%%%%%%%%%%%%%%%%%%%%%%%%%%%%%%%%%%%%%%%%%%%%%%%%%%%%%%%%%%%%%%%%%%%%%%%%%%%%%%%%%%%%%%%%%%%

\subsubsection{Vérifier si un itineraireflame est un circuit}

\raggedright

\begin{lstlisting}
unsigned int is_circuit(struct ItineraireFlame_t *way)
{
    assert(way != NULL);

    //$T_1$
    if (way->indicator == 0)
        return 0;

    //$T_2$
    for (int i = 0; i < way->indicator; i++)
    {
        //$T_{2.1}$
        if(way->map[i] == way->departure || way->map[i] == way->arrival)
            return 1;

        //$T_{2.2}$
        for(int y=i-1; y>=0; y--)
        {
            //$T_{2.2.1}$
            if(way->map[y] == way->map[i])
                return 1;
        }
    }

    //$T_3$
    return 0;
}
\end{lstlisting}

$T(.) = T_1 + T_2 + T_3$

\begin{description}
    \item[$T_1 = 1$] 
    \item[$T_2 = (T_{2.1} + T_{2.2})*indicator$]
    \indent
    \begin{description}
        \item[$T_{2.1} = 1$] 
        \item[$T_{2.2} = y*T_{2.2.1}=indicator$] 
    \end{description} 
    \item[$T_3 = 1$] 
\end{description}

Ainsi:
$T_2=(1 + indicator)*indicator \land 0 \le indicator \le ARRAY\_SIZE$
\newline
$T_2=ARRAY\_SIZE^2 + ARRAY\_SIZE$ 
\newline
$T(.) = ARRAY\_SIZE^2 + ARRAY\_SIZE + 2$

Conclusion: $T(.) \in O(ARRAY\_SIZE^2)$


%%%%%%%%%%%%%%%%%%%%%%%%%%%%%%%%%%%%%%%%%%%%%%%%%%%%%%%%%%%%%%%%%%%%%%%%%%%%%%%%%%%%%%%%%%%%%%%%
%%%%%%%%%%%%%%%%%%%%%%%%%%%%%%%%%%%%%%%%%%%%%%%%%%%%%%%%%%%%%%%%%%%%%%%%%%%%%%%%%%%%%%%%%%%%%%%%
%%%%%%%%%%%%%%%%%%%%%%%%%%%%%%%%%%%%%%%%%%%%%%%%%%%%%%%%%%%%%%%%%%%%%%%%%%%%%%%%%%%%%%%%%%%%%%%%

\subsubsection{Compter le nombre de region}

\begin{lstlisting}
unsigned int count_region(struct ItineraireFlame_t *way)
{
    assert(way != NULL);

    Region *tmp[ARRAY_SIZE];
    unsigned int is_counted, counter = 2;

    tmp[0] = way->departure;
    tmp[1] = way->arrival;

    for (int i = 0; i < way->indicator; i++)
    {
        is_counted = 0;
        for(int y = counter-1; y >= 0 && !is_counted; y--)
            is_counted = (tmp[y] == way->map[i]);
        
        if(!is_counted)
            tmp[counter++] = way->map[i];
    }
    
    return counter;
}
\end{lstlisting}

Par le même processus que celui de précédent on obtient: $T(.) \in O(ARRAY\_SIZE^2)$

%%%%%%%%%%%%%%%%%%%%%%%%%%%%%%%%%%%%%%%%%%%%%%%%%%%%%%%%%%%%%%%%%%%%%%%%%%%%%%%%%%%%%%%%%%%%%%%%
%%%%%%%%%%%%%%%%%%%%%%%%%%%%%%%%%%%%%%%%%%%%%%%%%%%%%%%%%%%%%%%%%%%%%%%%%%%%%%%%%%%%%%%%%%%%%%%%
%%%%%%%%%%%%%%%%%%%%%%%%%%%%%%%%%%%%%%%%%%%%%%%%%%%%%%%%%%%%%%%%%%%%%%%%%%%%%%%%%%%%%%%%%%%%%%%%

\subsubsection{Ajouter une région}

\begin{lstlisting}
unsigned int add_region(struct ItineraireFlame_t *way, Region *region)
{
    assert(way != NULL && region != NULL);
    
    //$T_1$
    if (way->indicator >= ARRAY_SIZE)
        return 0;

    //$T_2$
    way->map[way->indicator++] = region;

    return 1;
}
\end{lstlisting}

\raggedright
$T(.) = T_1 + T_2 = 2$

Conclusion: $T(.) \in O(1)$

%%%%%%%%%%%%%%%%%%%%%%%%%%%%%%%%%%%%%%%%%%%%%%%%%%%%%%%%%%%%%%%%%%%%%%%%%%%%%%%%%%%%%%%%%%%%%%%%
%%%%%%%%%%%%%%%%%%%%%%%%%%%%%%%%%%%%%%%%%%%%%%%%%%%%%%%%%%%%%%%%%%%%%%%%%%%%%%%%%%%%%%%%%%%%%%%%
%%%%%%%%%%%%%%%%%%%%%%%%%%%%%%%%%%%%%%%%%%%%%%%%%%%%%%%%%%%%%%%%%%%%%%%%%%%%%%%%%%%%%%%%%%%%%%%%

\subsubsection{Supprimer une région de l'itineraireflame}

\begin{lstlisting}
void remove_region(struct ItineraireFlame_t *way, Region *region)
{
    assert(way != NULL && region != NULL);

    //$T_1$
    int i = way->indicator;

    //$T_2$
    while (i >= 0)
    {
        //$T_{2.1}$
        if (way->map[i] == region)
        {
            //$T_{2.1.1}$
            for (int y = i+1; i < way->indicator; i++)
                way->map[y-1] = way->map[y];
            
            //$T_{2.1.2}$
            way->indicator--;
        }

        //$T_{2.2}$
        i--;
    }
}
\end{lstlisting}
$T(.)=T_1 + T_2$

\begin{description}
    \item[$T_1 = 1$] 
    \item[$T_2 = indicator*(T_{2.1} + T_{2.2})$]
        \indent 
        \begin{description}
            \item[$T_{2.1} = T_{2.1.1} + T_{2.1.2}$]
                \indent
                \begin{description}
                    \item[$T_{2.1.1} = indicator$] 
                    \item[$T_{2.1.2} = 1$] 
                \end{description}
            \item[$T_{2.2} = 1$]
        \end{description} 
\end{description}

Ainsi:
\newline$\rightarrow T_{2.1} = indicator + 1 \land T_{2.2} = 1$
\newline$\rightarrow T_2 = indicator*(indicator + 2)$ 
\newline$\rightarrow T(.) = indicator^2 + indicator + 1 \land 0 \le indicator \le ARRAY\_SIZE$ 
\newline$\rightarrow T(.) = ARRAY\_SIZE^2 + ARRAY\_SIZE + 1$

Conclusion: $T(.) \in O(ARRAY\_SIZE^2)$